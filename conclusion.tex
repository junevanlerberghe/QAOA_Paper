\section{Conclusion}
The Quantum Approximate Optimization Algorithm (QAOA) presents a promising approach to solving combinatorial optimization problems on NISQ devices. By leveraging the power of hybrid quantum-classical methods, QAOA can outperform classical algorithms such as Goemans-Williamson. The approximation ratio of QAOA increases as the number of layers increases rather than system size, which makes it a promising algorithm as the hardware improves. Applications in areas such as logistics, network optimization, and portfolio management highlight its versatility and usefulness.

Many variations and improvements have been made to QAOA, including warm start, recursive, and CVaR, which have reduced the complexity and number of layers required while increasing the approximation ratio. However, there are still challenges regarding the algorithm's performance on current devices compared to classical algorithms. NISQ devices are still too small and noisy to accurately find solutions that are better than the classical algorithms. 

There is much future work to be done regarding quantum hardware, error correction and improvements on QAOA to demonstrate its full potential. These developments will bring us closer to achieving quantum advantage and the ability to effectively approximate difficult optimization problems.
