\section{Introduction}
Optimization problems have many important applications in logistics, finance, and machine learning. Classical approaches to combinatorial optimization problems, such as Quadratic Unconstrained Binary Optimization (QUBO) and MaxCut, face significant challenges due to their NP-hard nature. Approximation algorithms have become widely researched, and there is growing interest in how quantum computers can help solve these problems.

Today's quantum hardware is considered Noisy Intermediate-Scale Quantum (NISQ), meaning they have a limited number of qubits and are noisy \cite{32preskill2018nisq}. Due to this, there is wide interest in developing algorithms that can run on these noisy computers. Variational Quantum Algorithms (VQAs) have emerged as a promising approach for using near-term devices due to their hybrid quantum-classical nature and shallow circuit depth \cite{vqa2021}. 

An example of a VQA is the Quantum Approximate Optimization Algorithm (QAOA), first proposed by Fahri, Goldstone, and Gutmann in 2014 \cite{farhiQAOA}. QAOA finds approximate solutions to combinatorial optimization problems by encoding the problem into a Hamiltonian, which is part of the quantum circuit. The other part of this circuit is a simpler "mixer" Hamiltonian. These Hamiltonians have parameters that are optimized classically. The performance of combinatorial optimization algorithms such as QAOA is measured by the approximation ratio: $\alpha = C_{QAOA}/C_{max}$ where $C_{QAOA}$ is the cost of the solution found by QAOA, and $C_{max}$ is the cost of the optimal solution \cite{review2024}. Theoretically, this approximation ratio increases with the number of layers in the QAOA circuit, p, making QAOA a promising candidate for optimization problems. 

The paper is organized as follows. Section 2 is the Background section, which includes all the necessary information to understand how QAOA works. It will explain the MaxCut problem, QUBO problems, variational quantum algorithms, and the quantum adiabatic algorithm (QAA). Section 3 explains QAOA, with a high-level overview and step-by-step explanation of how the algorithm works. Section 4 is an analysis covering the performance of QAOA (theoretically, simulated, and experimentally) and how it compares to the performance of classical algorithms. 
